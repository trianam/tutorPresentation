%This work is licensed under the Creative Commons
%Attribution-ShareAlike 4.0 International License. To view a copy of
%this license, visit http://creativecommons.org/licenses/by-sa/4.0/ or
%send a letter to Creative Commons, PO Box 1866, Mountain View, CA
%94042, USA.

%\documentclass[gray,handout, pdftex, 11pt]{beamer}
%\documentclass[handout, pdftex, 11pt]{beamer}

\documentclass[pdftex, 11pt]{beamer}

\usepackage[utf8]{inputenc}
\usepackage[T1]{fontenc}
\usepackage{lmodern}
\usepackage{microtype}
\usepackage[italian]{babel}
\usepackage{graphicx}
\usepackage{forloop}
\usepackage{tikz}

\definecolor{links}{HTML}{2A1B81}
\hypersetup{colorlinks,linkcolor=links,urlcolor=links}

\definecolor{links}{HTML}{2A1B81}
\hypersetup{colorlinks,linkcolor=,urlcolor=links}


\mode<presentation>{
  %-------------------------1
  \usetheme{Boadilla}
  \usecolortheme{beaver}
  %-------------------------1
  %-------------------------2
  %\usetheme{Goettingen}
  %\usecolortheme{sidebartab}
  %-------------------------2
  %\useoutertheme[right]{sidebar}
  %\usefonttheme{default}
  \setbeamercovered{transparent}
  %\setbeameroption{show notes on second screen=right}
  \setbeamertemplate{navigation symbols}{}
  \setbeamertemplate{footline}{}

  \bibliographystyle{abbrv}  
  %\renewcommand\bibfont{\scriptsize}
  \setbeamertemplate{bibliography item}{\textbullet}
  \setbeamertemplate{itemize item}{\checkmark}
  \setbeamertemplate{itemize subitem}{-}
  \setbeamertemplate{enumerate items}[default]
  \setbeamertemplate{sections/subsections in toc}[square]
}

\usebackgroundtemplate
{
  \begin{tikzpicture}
    \node[opacity=0.04] {\includegraphics[scale=1]{img/logoUnifi.png}};
%    \node[opacity=0.04] {\includegraphics[scale=1]{img/logoUnifi.eps}};
  \end{tikzpicture}
}

\begin{document}

\title[Piattaforma Moodle]{\textbf{Piattaforma Moodle e SOL}}
\date[28/09/15]{\flushright 28 settembre 2015}
\subtitle{Tutorato di informatica}
\institute[Uni. Firenze]{
  \includegraphics[width=2cm]{img/logoUnifi.eps}\\
%  \includegraphics[width=2cm]{img/logoUnifi.png}\\
  Universit\`a degli Studi di Firenze
}

\author[Stefano Martina]{
  Stefano \textsc{Martina}\\
  \href{mailto:stefano.martina@stud.unifi.it}{stefano.martina@stud.unifi.it}
}

\titlegraphic{
  \tiny
  \href{http://creativecommons.org/licenses/by-sa/4.0/}{\includegraphics[width=1cm]{img/logoCC.png}}
  This work is licensed under a
  \href{http://creativecommons.org/licenses/by-sa/4.0/}{Creative
    Commons Attribution-ShareAlike 4.0 International License}.
}

\begin{frame}[plain]
  \titlepage
\end{frame}

\begin{frame}
  \frametitle{Piattaforma Moodle (E-Learning)}
  \begin{itemize}
  \item La piattaforma \alert{Moodle} è il principale strumento usato dai
    professori per fornire \alert{materiale} e \alert{notizie}
    riguardo il corso
    \pause
  \item \`E accessibile all'indirizzo: \url{http://e-l.unifi.it/}
    \pause
  \item Permette di \alert{iscriversi} ai corsi usando una
    \alert{chiave} data in classe dal professore
    \pause
  \item \`E necessario \alert{autenticarsi} con le credenziali di
    accesso dell'universit\`a
  \end{itemize}
\end{frame}

\newcounter{screenNum}
\forloop{screenNum}{1}{\value{screenNum} < 9}{
  \begin{frame}
    \begin{center}
      \includegraphics[width=\textwidth]{img/screen\arabic{screenNum}.png}
    \end{center}
  \end{frame}
}

\begin{frame}
  \frametitle{Piattaforma SOL (Servizi On Line)}
  \begin{itemize}
  \item La piattaforma \alert{SOL} \`e lo strumento di gestione e
    controllo delle pratiche \alert{burocratiche} dell'universit\`a
    \pause
  \item \`E accessibile all'indirizzo \url{http://sol.unifi.it}
    \pause
  \item Permette di:
    \pause
    \begin{itemize}
    \item \alert{Registrarsi} per gli \alert{esami} (ricordatevi di
      non perdere la finestra di apertura dell'\alert{appello})
      \pause
    \item \alert{Verbalizzare} gli \alert{esami}
      \pause
    \item Stampare i \alert{bollettini} per il pagamento delle tasse
      \pause
    \item Controllare i dati \alert{anagrafici}, le
      \alert{iscrizioni}, e lo stato dei \alert{pagamenti} delle
      tasse, il \alert{piano di studi}, etc\dots
      \pause
    \item Reperire le \alert{licenze} software messe a disposizione
      per gli utenti
      \pause 
    \end{itemize}
  \item \`E necessario \alert{autenticarsi} con le credenziali di
    accesso dell'universit\`a
  \end{itemize}
\end{frame}

\forloop{screenNum}{1}{\value{screenNum} < 3}{
  \begin{frame}
    \begin{center}
      \includegraphics[width=\textwidth]{img/screenSol\arabic{screenNum}.png}
    \end{center}
  \end{frame}
}
\end{document}